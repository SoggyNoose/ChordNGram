\documentclass{article}
\usepackage[paperheight=11in,paperwidth=8.5in,margin=1in]{geometry}
\usepackage{setspace, fancyhdr}
\usepackage{mdwlist}
\usepackage{graphicx}
\usepackage{amsmath}
\pagestyle{fancy}

\setlength{\headheight}{15.2pt}
\setlength{\headsep}{12pt}

\begin{document}

\fancyhf{}
\lhead{Chordal Analysis}
\chead{Final Report}
\rhead{\today}

\begin{titlepage}
\begin{center}
{\huge Genre Classification by Chord Progression}\\[2cm]
{\Large Final Report}\\[2cm]
{\large \today}\\[2cm]
\emph{Team Members:}\\
Michael \uppercase{Eaton}\\
Samuel \uppercase{Kim}\\
\end{center}
\end{titlepage}

\tableofcontents
\newpage

\section{Introduction}

\newpage

\section{Chord Progressions}
Western music is based heavily on harmoic relations between notes 
that fall within major and minor scales. Each of these relations can be 
defined as a \emph{chord}.  For the purposes of this project, a chord 
is assumed to be at least a triad (it contains three or more notes).

	\subsection{Chord Construction}
	
		\subsubsection{Majors and Minors}
		All chords considered by this project fall into two groups:  major 
		and minor chords.  A major chord is built from the root and adding 
		the note that is four half-steps up (this is the \emph{third} of the 
		chord) and the note that is seven half-steps up (this is the \emph{fifth}) 
		of the chord).

		\begin{figure}[htb]
		%\includegraphics{cMajorPiano}
		\caption{A  C-Major chord, as shown on a piano}
		\end{figure} 

		A minor chord is similar to a major chord, except that the \emph{third} 
		is three half steps up from the root, rather than four.

		\subsubsection{Qualities}
		Chords are not simply limited to the three notes that build their basic 
		structure.  Any additional notes that are added to them will give them 
		\emph{qualities}. To simplify data collection and allow for more meaningful 
		results, this project has only considered the following qualities:

		\begin{itemize}
			\item Major Seventh:  The base chord has added the note twelve half 
				steps up from the root.\\
				\emph{Example:  adding a B to a C chord}
			\item Minor Seventh:  The base chord has added the note eleven half 
				steps up from the root.\\  
				\emph{Example:  adding a Bflat to a C chord}
			\item Suspended:  The base chord has added the note two half steps 
				up or five half steps up from the root. \\  
				\emph{Example:  Adding a D or F to a C chord}\\
				\emph{Note:  This is a blanket definition 
				for the 4, 2, Suspended 4, and Suspended 2 qualities.  This decision 
				was made because they are similar in overall tone and occurence}
		\end{itemize}

	\subsection{Chord Progressions}
	Given these chords, a \emph{chord progression} is the series of chords 
	that make up a given piece of music.  Given the key of a piece of music, 
	one notates the interval of each of the chords from the key's root.  
	Notation is in roman numerals, with major chords being denoted by capital 
	I and Vs, while minor chords have lower case i's and v's.

		\subsubsection{An Example}
		The song \emph{The Crane Wife Part III} by The Decemberists is a song 
		in the key of D.  The flow of the song's chords is:  D, A, G, D, A, G, D, A, G.  
		This is notated as:\\

		\begin{center}
		\emph{I - V - IV - I - V - IV - I - V - IV}
		\end{center}

		\subsubsection{Application}
		The chord progression of a song will give a musician an idea of the general 
		tones and harmonies in a specific piece of music, and the order they 
		come in.  With this information and an idea of the melody, it is simply 
		a matter of practice to learn and play a song.

		Because this is such a powerful tool for describing a song, we hypothesize 
		that the chord progression of a song can be used to accurately predict a 
		song's genre of music.  Because of the linear nature of a chord progression, 
		it makes it an ideal candidate for n-gram analysis.
  
\newpage

\section{N-grams}

In the field of probability, an \emph{n-gram} is a sequence of $n$ terms. These can be
words, syllables, numbers, or abstract symbols. An \emph{n-gram model} is a method
for predicting the next item in a sequence as a \emph{(n-1)-order Markov model}. 

For our system, the symbols are the chords. We are also not aiming to predict the
next chord in a sequence (although that would make for an interesting concept in
itself). Instead, we aim to find the probability for a given sequence to occur, given
a particular genre. We can do this by providing training data for the model to reference.

As for the actual model, we decided that the best way to represent the problem is
that each genre would be its own model. Then given a sequence and training data,
we could score that sequence for each model and compare results. 

For a particular model, we build a table of \emph{bigrams} or 2-gram -- that is, we group
a sequence into pairs. Chord progressions are commonly in groups of four, so it 
seems intuitive to group them by fours. However,
given the number of chord and quality combinations, it was impractical to gather
enough data during the time of the project. Therefore, we chose bigrams for
simplicity. In addition to the bigram, each n-gram has a \emph{context}, which
is simply defined to be the first $n-1$ symbols in the sequence.

An example of the table is shown in Figures \ref{tab:bigram} and \ref{tab:context}.

	\begin{figure}[h]

	\begin{center}
		\begin{tabular}{l | l | l | l | l | l | l | l}
•		bigram &	I-I 	& I-i	& I-II	& ...	& VM7-isus	& VM7-II	& ... \\ \hline
		counts & 17	& 0	& 7	& ...	& 2			& 1		& ... \\
		\end{tabular}•
	\end{center}
	\caption{The bigram table (numbers are not representative of actual data)}
	\label{tab:bigram}

	
	\begin{center}
		\begin{tabular}{l|l|l|l|l|l|l|l}
•		context	&	I 	&	II	&	III	&	...	&	vi	&	vii	&	...	\\[0.3em] \hline
		counts	&	78	&	97	&	103	&	...	&	43	&	29	&	...	\\
		\end{tabular}•
	\end{center}•
	\caption{The table of contexts (numbers are not representative of actual data)}
	\label{tab:context}
	\end{figure}•

To find the probability of a bigram occuring, we can simply take

	\[
	P(bigram) = \frac{\text{count of the bigram}}{\text{count of the context}}
	\]

To score a sequence such as \emph{I - V - IV - I - V - IV - I - V - IV}, we can calculate

	\[
	P(sequence) = P(I - V - IV) * P(V - IV - I) * P(IV - I - V) * P(I - V - IV) * ...
	\]

Problems can arise in the case that the count of the context never occured
in the training data,
as we will divide by zero. However, if the context hasn't been seen, then
trivially the bigram hasn't either. So we treat the pair as if we trained the
model with just the bigram. Therefore, the counts of both are simply 1.

Further problems arise when a bigram hasn't been encountered, yet the 
context has. This will cause the probably to become 0, and likewise the
score of the entire sequence. To solve this, we implement a technique 
known as \emph{LaPlace Add One} technique. It is rather simple -- we add
1 to the count of every other bigram in the table. With enough data, this
adds an insignificant amount to the scores of every other bigram,
while avoiding a result of 0.

\newpage

\section{Program Design}
The program has three major components:  A seralizeable n-gram class,
a main controller class, and a bi-gram class.  It imports data from a text 
file for usage in the bigrams % Incomplete?

	\subsection{The Bi-gram}
	The bi-gram class is a simple container class that holds two strings, each 
	representing a different chord symbol with a quality.


		\subsection{The N-gram}
		The n-gram class is a container class with logic for importing additional 
		data into the set.  It holds a Map of integers, keyed to bi-grams.  When 
		data is imported, it is imported as a series of chord symbols parsed 
		from a file via regular expressions.  Bi-grams are generated from this 
		string, counted, and added to the map.  When the program closes, each 
		N-gram object is serialized and saved to a file, where it is reopened and 
		loaded at the next start of the program.

		This class also contains the logic for finding the probability of a given 
		series of chord symbols.  Missing symbols are handled via Laplace Add One.

	\subsection{The Controller}
	The controller handles the various user-interactions the program requires.  
	It provides logic and options for direction of inputs for data import, an 
	interface for determining the genre of a song, and options for resetting 
	the program's data.

\newpage

\section{Results}

The system was trained by entering in a variety of different progressions,
and determining the results. The outputs are as follows:

\begin{itemize}
\item \emph{I - IV - V - V} This is a very common progression often
	found in classic rock. The model seems to agree. \\
	\emph{Probability of being Classic is 97.52114319043453%} \\
	\emph{Probability of being Indie is 2.4788568095654715%} \\
\item \emph{vi - Vsus - IV - Vsus - I - Vsus}. This was an excerpt
	from a progression for the Indie progression set. \\
	\emph{Probability of being Classic is 99.04968775454792%} \\
	\emph{Probability of being Indie is 0.9503122454520773%} \\
	The result can be attributed to the fact that many of the
	chord transitions were never seen in classic rock (as
	sus chords in general are rare in the genre)
\item \emph{V - IV - I - V}. Yet another excerpt from an Indie
	song.
	\emph{Probability of being Classic is 11.863083955847921% \\
	Probability of being Indie is 88.13691604415207%} \\
\item \emph{I - V - vii - vii}. This is a progression that is
	neither Classic nor Indie rock, but is rather Alternative
	rock.
	\emph{Probability of being Classic is 64.02569593147753% \\
	Probability of being Indie is 35.974304068522486%}
\item \emph{vi - I - III - IV - V}. Taken from a bluegrass track. \\
	\emph{Probability of being Classic is 86.02684395679195% \\
	Probability of being Indie is 13.973156043208046%}
\item \emph{i - III - VII - IV}. Taken from an Indie song not
	included in the training set. \\
	\emph{Probability of being Classic is 60.796139927623635% \\
	Probability of being Indie is 39.20386007237636%}
\item \emph{vi - I - IV - I}. Taken from a classic rock song
	not included in the training set. \\
	\emph{Probability of being Classic is 64.47975397232189% \\
	Probability of being Indie is 35.52024602767811%}

\newpage

\section{Conclusion}

% Results generally show classic rock beating out Indie. This is
% primarily due to the fact that Indie rock artists use chords
% and chord transitions typically not seen in Classic rock.
% This causes the Laplace Add One to greatly favor Classic rock.
% Can be alleviated by using absolute discounting, getting more
% data, using some sort of composite scoring (i.e. introducing
% other factors, etc.)

% We should also mention something about challenges faced,
% flaws with the system, possible future features, etc.

\end{document}
