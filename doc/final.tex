\documentclass{article}
\usepackage[paperheight=11in,paperwidth=8.5in,margin=1in]{geometry}
\usepackage{setspace, fancyhdr}
\usepackage{mdwlist}
\usepackage{graphicx}
\pagestyle{fancy}

\setlength{\headheight}{15.2pt}
\setlength{\headsep}{12pt}

\begin{document}

\fancyhf{}
\lhead{Chordal Analysis}
\chead{Final Report}
\rhead{\today}

\begin{titlepage}
\begin{center}
{\huge Genera Classification by Chord Progression}\\[2cm]
{\Large Final Report}\\[2cm]
{\large \today}\\[2cm]
\emph{Team Members:}\\
Michael \uppercase{Eaton}\\
Samuel \uppercase{Kim}\\
\end{center}
\end{titlepage}

\tableofcontents
\newpage

\section{Introduction}

\newpage

\section{Chord Progressions}
Western music is based heavily on harmoic relations between notes that fall within major and minor scales. Each of these relations can be defined as a \emph{chord}.  For the purposes of this project, a chord is assumed to be at least a triad (it contains three or more notes).
\subsection{Chord Construction}
\subsubsection{Majors and Minors}
All chords considered by this project fall into two groups:  Major and minor chords.  A major chord is built from the root and adding the note that is four half-steps up (this is the \emph{third} of the chord) and the note that is seven half-steps up (this is the \emph{fifth}) of the chord).\\
\begin{figure}[htb]
\includegraphics{cMajorPiano}
\caption{A  C-Major chord, as shown on a piano}
\end{figure} \\
A minor chord is similar to a major chord, except that the \emph{third} is three half steps up from the root, rather than four.
\subsubsection{Qualities}
Chords are not simply limited to the three notes that build their basic structure.  Any additional notes that are added to them will give them \emph{qualities}.  To simplify data collection and allow for more meaningful results, this project has only considered the following qualities:
\begin{itemize}
	\item Major Seventh:  The base chord has added the note twelve half steps up from the root.\\  \emph{Example:  adding a B to a C chord}
	\item Minor Seventh:  The base chord has added the note eleven half steps up from the root.\\  \emph{Example:  adding a Bflat to a C chord}
	\item Suspended:  The base chord has added the note two half steps up or five half steps up from the root. \\  \emph{Example:  Adding a D or F to a C chord}\\\emph{Note:  This is a blanket definition for the 4, 2, Suspended 4, and Suspended 2 qualities.  This decision was made because they are similar in overall tone and occurence}
\end{itemize}
\subsection{Chord Progressions}
Given these chords, a \emph{chord progression} is the series of chords that make up a given piece of music.  Given the key of a piece of music, one notates the interval of each of the chords from the key's root.  Notation is in roman numerals, with major chords being denoted by capital I and Vs, while minor chords have lower case i's and v's.
\subsubsection{An Example}
The song \emph{The Crane Wife Part III} by The Decemberists is a song in the key of D.  The flow of the song's chords is:  D, A, G, D, A, G, D, A, G.  This is notated as:\\
\begin{center}
\emph{I - V - IV - I - V - IV - I - V - IV}
\end{center}
\subsubsection{Application}
The chord progression of a song will give a musician an idea of the general tones and harmonies in a specific piece of music, and the order they come in.  With this information and an idea of the melody, it is simply a matter of practice to learn and play a song.\\
Because this is such a powerful tool for describing a song, we hypothesize that the chord progression of a song can be used to accurately predict a song's genera of music.  Because of the linear nature of a chord progression, it makes it an ideal candidate for n-gram analysis.
  
\newpage

\section{N-grams}

\newpage

\section{Program Design}
The program has three major components:  A seralizeable n-gram class, a main controller class, and a bi-gram class.  It imports data from a text file for usage in the bigrams/
\subsection{The Bi-gram}
The bi-gram class is a simple container class that holds two strings, each representing a different chord symbol with a quality.
\subsection{The N-gram}
The n-gram class is a container class with logic for importing additional data into the set.  It holds a Map of integers, keyed to bi-grams.  When data is imported, it is imported as a series of chord symbols parsed from a file via regular expressions.  Bi-grams are generated from this string, counted, and added to the map.  When the program closes, each N-gram object is serialized and saved to a file, where it is reopened and loaded at the next start of the program.\\
This class also contains the logic for finding the probability of a given series of chord symbols.  Missing symbols are handled via Laplace Add One.
\subsection{The Controller}
The controller handles the various user-interactions the program requires.  It provides logic and options for direction of inputs for data import, an interface for determining the genera of a song, and options for resetting the program's data.

\newpage

\section{Results}

\newpage

\section{Conclusion}

\end{document}
